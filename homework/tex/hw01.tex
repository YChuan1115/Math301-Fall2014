% Example LaTeX document for GP111 - note % sign indicates a comment
\documentclass[12pt,reqno]{amsart}
\usepackage[top=1.5cm, left=1.5cm,right=1.5cm,bottom=1.5cm]{geometry}

\usepackage{amsmath}
\usepackage{amssymb}
\usepackage{color,hyperref,enumerate,multicol}
\definecolor{darkblue}{rgb}{0.0,0.0,0.3}
\hypersetup{colorlinks,breaklinks,
            linkcolor=darkblue,urlcolor=darkblue,
            anchorcolor=darkblue,citecolor=darkblue}
            
\usepackage{algorithm}
\usepackage{algorithmic}
\pagestyle{empty}
\newcommand{\N}{\ensuremath{\mathbb{N}}}
\newcommand{\Z}{\ensuremath{\mathbb{Z}}}
\newcommand{\R}{\ensuremath{\mathbb{R}}}
\newcommand{\meet}{\ensuremath{\wedge}}
\newcommand{\Meet}{\ensuremath{\bigwedge}}
\newcommand{\join}{\ensuremath{\vee}}
\renewcommand{\emptyset}{\ensuremath{\varnothing}}

\begin{document}
\thispagestyle{empty}

\noindent \textbf{Math 301} \hskip4cm {\bf Homework 1} \hfill {\bf Fall 2014}
\vskip1cm
\noindent {\bf Chapter 1:}  1cd, 2bd, 3, 7, 20b, 22cd, 24bc(de), 25d, 28.  \\
{\bf Due date:} Friday, 8/29

\medskip

\begin{enumerate}[{\bf 1.}]

%% 1 %%%%%%%%%%%%%%%%%%%%%%%%%%%%%%%%%%%%%%%%%%%%%%%%
\item[{\bf 1.}]
Suppose that
\begin{align*}
A & = \{ x : x \in \mathbb N \text{ and } x \text{ is even} \}, \\
B & = \{x : x \in \mathbb N \text{ and } x \text{ is prime}\}, \\
C & = \{ x : x \in \mathbb N \text{ and } x \text{ is a multiple of 5}\}.
\end{align*}
Describe each of the following sets. 
\begin{multicols}{2}
\begin{enumerate}

\item
$A \cap B$

\item
$B \cap C$

\item
$A \cup B$

\item
$A \cap (B \cup C)$

\end{enumerate}
\end{multicols}

\medskip  

%% 2 %%%%%%%%%%%%%%%%%%%%%%%%%%%%%%%%%%%%%%%%%%%%%%%%
\item[{\bf 2.}]
If $A = \{ a, b, c \}$, $B = \{ 1, 2, 3 \}$, $C = \{ x \}$, and 
$D = \emptyset$, list all of the elements in each of the following sets. 
\begin{multicols}{2}
\begin{enumerate}

\item
$A \times B$

\item
$B \times A$

\item
$A \times B \times C$

\item
$A \times D$

\end{enumerate}
\end{multicols}

\medskip  

%% 3 %%%%%%%%%%%%%%%%%%%%%%%%%%%%%%%%%%%%%%%%%%%%%%%%
\item[{\bf 3.}]
Find an example of two nonempty sets $A$ and $B$ for which $A \times B = B \times A$.

\medskip

 
%% 7 %%%%%%%%%%%%%%%%%%%%%%%%%%%%%%%%%%%%%%%%%%%%%%%%
\item[{\bf 7.}]
Prove $A \cap (B \cup C) = (A \cap B) \cup (A \cap C)$.

\medskip

% 20 %%%%%%%%%%%%%%%%%%%%%%%%%%%%%%%%%%%%%%%%%%%%%%%%%
\item[{\bf 20.}]
\begin{enumerate}
  
\item
Define a function $f: {\mathbb N} \rightarrow {\mathbb N}$ that is one-to-one but not onto. 
 
\item
Define a function $f: {\mathbb N} \rightarrow {\mathbb N}$ that is onto but not one-to-one. 
 
\end{enumerate}

\medskip
 
% 22 %%%%%%%%%%%%%%%%%%%%%%%%%%%%%%%%%%%%%%%%%%%%%%%%%
\item[{\bf 22.}]
Let $f : A \rightarrow B$ and $g : B \rightarrow C$ be maps.
\begin{enumerate}
 
\item
If $f$ and $g$ are both one-to-one functions, show that $g \circ f$
is one-to-one. 
 
\item
If $g \circ f$ is onto, show that $g$ is onto.
 
\item
If $g \circ f$ is one-to-one, show that $f$ is one-to-one.
 
\item
If $g \circ f$ is one-to-one and $f$ is onto, show that $g$ is
one-to-one.
 
\item
If $g \circ f$ is onto and $g$ is one-to-one, show that $f$ is onto.
 
\end{enumerate}

\medskip

%% 24 %%%%%%%%%%%%%%%%%%%%%%%%%%%%%%%%%%%%%%%%%%%%%%%%%
\item[{\bf 24.}]
Let $f: X \rightarrow Y$ be a map with $A_1, A_2 \subset X$ and $B_1, B_2 \subset Y$. 
\begin{enumerate}
 
\item
Prove $f( A_1 \cup A_2 ) = f( A_1) \cup f( A_2 )$.
 
\item
Prove $f( A_1 \cap A_2 ) \subset f( A_1) \cap f( A_2 )$.  Give an example in which equality fails.
 
\item
Prove $f^{-1}( B_1 \cup B_2 ) = f^{-1}( B_1) \cup f^{-1}(B_2 )$, where
\[
f^{-1}(B) = \{ x \in X : f(x) \in B \}.
\]
 
\item
Prove $f^{-1}( B_1 \cap B_2 ) = f^{-1}( B_1) \cap f^{-1}( B_2 )$. 
 
\item
Prove $f^{-1}( Y \setminus B_1 ) = X \setminus f^{-1}( B_1)$.
 
\end{enumerate}

\medskip

%% 25 %%%%%%%%%%%%%%%%%%%%%%%%%%%%%%%%%%%%%%%%%%%%%%%%%%%%%%%%
\item[{\bf 25.}]
Determine whether or not the following relations are equivalence relations on the given set.  If the relation is an equivalence relation, describe the partition given by it.  If the relation is not an equivalence relation, state why it fails to be one.
\begin{multicols}{2}
\begin{enumerate}
 
\item
$x \sim y$ in ${\mathbb R}$ if $x \geq y$
 
\item
$m \sim n$ in ${\mathbb Z}$ if $mn > 0$
 
\item
$x \sim y$ in ${\mathbb R}$ if $|x - y| \leq 4$
 
\item
$m \sim n$ in ${\mathbb Z}$ if $m \equiv n \pmod{6}$
 
\end{enumerate}
\end{multicols}
 
\medskip

 
\item[{\bf 28.}]
Find the error in the following argument by providing a counterexample. ``The
reflexive property is redundant in the axioms for an equivalence relation.  If
$x \sim y$, then $y \sim x$ by the symmetric property.  Using the transitive
property, we can deduce that $x \sim x$.''
 
\end{enumerate}
\end{document}

