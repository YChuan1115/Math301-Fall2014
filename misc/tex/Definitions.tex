\documentclass[12pt]{article}
\usepackage{amsmath}
\usepackage{amssymb}
\usepackage{amsthm}
\usepackage{geometry}
\usepackage{fancyhdr}
\usepackage{scalefnt}
\usepackage{url}
\usepackage{tikz}
\pagestyle{fancy} \lhead{\bf Math 301} \chead{\bf }
\rhead{\bf Definitions} \lfoot{} \cfoot{\thepage} \rfoot{}
\renewcommand{\headrulewidth}{0.6pt}
\renewcommand{\footrulewidth}{0.6pt}
\setlength{\headwidth}{6.5in}
% Fuzz -------------------------------------------------------------------
\hfuzz2pt % Don't bother to report over-full boxes if over-edge is < 2pt
% Line spacing -----------------------------------------------------------
\newlength{\defbaselineskip}
\setlength{\defbaselineskip}{\baselineskip}
\newcommand{\setlinespacing}[1]
           {\setlength{\baselineskip}{#1 \defbaselineskip}}
\newcommand{\doublespacing}{\setlength{\baselineskip}%
                           {2.0 \defbaselineskip}}
\newcommand{\singlespacing}{\setlength{\baselineskip}{\defbaselineskip}}
\setlength{\textwidth}{6.5in} \setlength{\textheight}{9in}
\setlength{\oddsidemargin}{.1in}
\setlength{\evensidemargin}{.1in} \setlength{\voffset}{-.5in}
\setlength{\topmargin}{0pt}
\newcommand{\boldemph}[1]{#1}
\newcommand\<{\ensuremath{\langle}}
\renewcommand\>{\ensuremath{\rangle}}
\newcommand\meet{\ensuremath{\wedge}}
\newcommand\join{\ensuremath{\vee}}
\newcommand\bS{\ensuremath{\mathbf{S}}}
\newcommand\bA{\ensuremath{\mathbf{A}}}
\newcommand\bR{\ensuremath{\mathbf{R}}}
\begin{document}




Students must know the precise definitions of the following terms:
\begin{itemize}
\item Cartesian product
\item relation
\item function
\item operation
\item universe
\item arity of relation, function, or operation (e.g., nullary, unary, binary, ternary, $n$-ary)
\item $n$-ary relation on a set $X$ (notation: $\rho \subseteq X^n$)
\item $n$-ary function from set $X$ to set $Y$ (notation: $f: X^n \rightarrow Y$)
\item $n$-ary operation on a set $X$ (notation: $f: X^n \rightarrow X$)
\item interpretation of $n$-ary operation as special kind of $(n+1)$-ary relation
\item properties binary relations might satisfy: \boldemph{reflexive},
  \boldemph{(anti)symmetric}, \boldemph{transitive}
\item properties functions might satisfy: onto, one-to-one, bijective
\item properties operations might satisfy: commutative, associative, idempotent
\item equivalence relation
\item equivalence class
\item partition
\item congruence modulo $n$
\item partial order
\item partially ordered set (poset), $\<P, \preceq\>$
\item total order
\item totally ordered set
\item well-ordered set
\item common divisor
\item greatest common divisor
\item common multiple
\item least common multiple
\item relatively prime
\item prime number
\item prime factorization
\item power set
\item relational structure, $\<A, \mathcal{R}\>$, with universe $A$ and relations $\mathcal{R}$ 
\item algebraic structure, $\<A, \mathcal{F}\>$, with universe $A$ and operations $\mathcal{F}$ 
\item examples of relational structures (e.g., poset, graph)\footnote{\label{note1}\emph{Many} more examples at \url{http://www.math.chapman.edu/~jipsen/structures/doku.php/index.html}}
\item examples of algebraic structures (e.g., magma, semigroup, monoid, group)\footnotemark[\ref{note1}]
\item identity element
\item inverse operation
\item abelian group
\item Cayley table
\item finite group
\item subgroup, proper subgroup, trivial subgroup
\item order (of a group or subgroup)
\item order (of a group element)
\item $g^n$ and $g^{-n}$ (for $g$ an element of a multiplicative group)
\item $ng$ and $-ng$ (for $g$ an element of an additive group)
\item cyclic group
\item generator (of a cyclic group)
\item generators (of a group)
\item symmetry, rigid motion
\item permutation (and two ways to write them)
\item cycle
\item length of a cycle
\item transposition
\item even, odd (permutation)
\item examples of groups: $\mathbb{Z}_n$, $U(n)$, $S_n$, $A_n$, $D_4$
\item \boldemph{upper bound} (of a subset of a poset, lattice, or join semilattice)
\item \boldemph{least upper bound} or \boldemph{supremum} or join
\item \boldemph{lower bound}
\item \boldemph{greatest lower bound} or \boldemph{infimum} or meet
\item \boldemph{lattice}, $\<L, \meet, \join\>$
\item \boldemph{semilattice}, $\<S, \cdot\>$
\item \boldemph{meet semilattice}, $\<S, \meet\>$
\item \boldemph{join semilattice}, $\<S, \join\>$
\item \boldemph{join} (of elements), $a\join b$
\item \boldemph{meet} (of elements), $a\meet b$
\item \boldemph{join} (of a subset), $\bigvee T$
\item \boldemph{meet} (of a subset), $\bigwedge T$
\item \boldemph{largest element} (of a poset; need not exist)
\item \boldemph{smallest element} (of a poset; need not exist)
\item order-preserving map
\item lattice homomorphism
\item \boldemph{left coset}
\item \boldemph{right coset}
\item \boldemph{coset representative}
\item \boldemph{index} (of a subgroup)
\item \boldemph{Euler} $\varphi$ \boldemph{function}
\item \boldemph{conjugate} elements of a group
\end{itemize}
\end{document}
