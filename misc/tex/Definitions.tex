\documentclass[12pt]{article}
\usepackage{amsmath}
\usepackage{amssymb}
\usepackage{amsthm}
\usepackage{geometry}
\usepackage{fancyhdr}
\usepackage{scalefnt}
\usepackage{tikz}
\pagestyle{fancy} \lhead{\bf Math 301} \chead{\bf }
\rhead{\bf Definitions} \lfoot{} \cfoot{\thepage} \rfoot{}
\renewcommand{\headrulewidth}{0.6pt}
\renewcommand{\footrulewidth}{0.6pt}
\setlength{\headwidth}{6.5in}
% Fuzz -------------------------------------------------------------------
\hfuzz2pt % Don't bother to report over-full boxes if over-edge is < 2pt
% Line spacing -----------------------------------------------------------
\newlength{\defbaselineskip}
\setlength{\defbaselineskip}{\baselineskip}
\newcommand{\setlinespacing}[1]
           {\setlength{\baselineskip}{#1 \defbaselineskip}}
\newcommand{\doublespacing}{\setlength{\baselineskip}%
                           {2.0 \defbaselineskip}}
\newcommand{\singlespacing}{\setlength{\baselineskip}{\defbaselineskip}}
\setlength{\textwidth}{6.5in} \setlength{\textheight}{9in}
\setlength{\oddsidemargin}{.1in}
\setlength{\evensidemargin}{.1in} \setlength{\voffset}{-.5in}
\setlength{\topmargin}{0pt}

\begin{document}




Students must know the precise definitions of the following terms:
\begin{itemize}
\item Cartesian product
\item relation
\item function
\item arity of a relation or function (e.g., nullary, unary, binary, ternary, $n$-ary)
\item onto, one-to-one, bijective function
\item equivalence relation
\item partition
\item equivalence class
\item congruence modulo $n$
\item partial order
\item partially ordered set
\item total order
\item totally ordered set
\item well-ordered set
\item common divisor
\item greatest common divisor
\item common multiple
\item least common multiple
\item relatively prime
\item prime number
\item prime factorization
\item power set
\item algebra (or algebraic structure)
\item universe
\item operation
\item associative and commutative properties (of binary operations)
\item magma
\item semigroup
\item monoid
\item group
\item identity element
\item inverse operation
\item abelian group
\item Cayley table
\item finite group
\item order (of a finite group)
\item $g^n$ and $g^{-n}$ (for $g$ an element of a multiplicative group)
\item $ng$ and $-ng$ (for $g$ an element of an additive group)
\item order (of a group element)
\item subgroup, proper subgroup, trivial subgroup
\item cyclic group
\item generator
\item symmetry, rigid motion
\item permutation (and two ways to denote a permutation)
\item cycle
\item length (of a cycle)
\item transposition
\item even, odd (permutation)
\item examples of groups: $\mathbb{Z}_n$, $U(n)$, $S_n$, $A_n$, $D_4$
\end{itemize}
\end{document}
