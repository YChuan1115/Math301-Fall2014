% Example LaTeX document for GP111 - note % sign indicates a comment
\documentclass[12pt,reqno]{amsart}
\usepackage[top=1.5cm, left=1.5cm,right=1.5cm,bottom=1.5cm]{geometry}
\renewcommand{\baselinestretch}{1.2}
\usepackage{amsmath}
\usepackage{amssymb}
\usepackage{color,hyperref,enumerate,multicol}
\definecolor{darkblue}{rgb}{0.0,0.0,0.3}
\hypersetup{colorlinks,breaklinks,
            linkcolor=darkblue,urlcolor=darkblue,
            anchorcolor=darkblue,citecolor=darkblue}
            
\usepackage{algorithm}
\usepackage{algorithmic}
\pagestyle{empty}
\newcommand{\N}{\ensuremath{\mathbb{N}}}
\newcommand{\Z}{\ensuremath{\mathbb{Z}}}
\newcommand{\R}{\ensuremath{\mathbb{R}}}
\newcommand{\meet}{\ensuremath{\wedge}}
\newcommand{\Meet}{\ensuremath{\bigwedge}}
\newcommand{\join}{\ensuremath{\vee}}
\renewcommand{\emptyset}{\ensuremath{\varnothing}}
\renewcommand{\subset}{\ensuremath{\subsetneq}}
\newcommand{\boldemph}{\emph}
\newcommand{\lcm}{\operatorname{lcm}}
\newcommand{\cis}{\ensuremath{\operatorname{cis}}}

\begin{document}
\thispagestyle{empty}

\noindent \textbf{Abstract Algebra} \hskip3cm {\bf Thomas Judson} \hfill {\bf Chapter 4 Exercises}
\medskip

\begin{enumerate}[{\bf 1.}]
\item
Prove or disprove each of the following statements.
\begin{enumerate}
 
 \item
$U(8)$ is cyclic.
 
 \item
All of the generators of ${\mathbb Z}_{60}$ are prime.
 
 \item
${\mathbb Q}$ is cyclic.
 
 \item
If every proper subgroup of a group $G$ is cyclic, then $G$ is a cyclic
group. 
%Changed the problem to read proper subgroup.  Suggested by A. Glesser. - TWJ 2/1/2012
 
 \item
A group with a finite number of subgroups is finite.
 
\end{enumerate}
 
  
\item
Find the order of each of the following elements.
\begin{multicols}{3}
\begin{enumerate}

\item
$5 \in {\mathbb Z}_{12}$

\item
$\sqrt{3} \in {\mathbb R}$
 
\item
$\sqrt{3} \in {\mathbb R}^\ast$
 
\item
$-i \in {\mathbb C}^\ast$

\item
72 in ${\mathbb Z}_{240}$
 
\item
312 in ${\mathbb Z}_{471}$
 
 \end{enumerate}
 \end{multicols}
  
\item
List all of the elements in each of the following subgroups.
\begin{enumerate}
 
 \item
The subgroup of ${\mathbb Z}$ generated by 7
 
 \item
The subgroup of ${\mathbb Z}_{24}$ generated by 15
 
 \item
All subgroups of ${\mathbb Z}_{12}$
 
 \item
All subgroups of ${\mathbb Z}_{60}$
 
 \item
All subgroups of ${\mathbb Z}_{13}$
 
 \item
All subgroups of ${\mathbb Z}_{48}$
 
 \item
The subgroup generated by 3  in $U(20)$
 
 \item
The subgroup generated by 5 in $U(18)$  %Changed 6 to 5 so that the problem would make sense.  Suggested by Aziz Azizov - TWJ 12/9/2010
 
 \item
The subgroup of ${\mathbb R}^\ast$ generated by 7
 
 \item
The subgroup of ${\mathbb C}^\ast$ generated by $i$ where $i^2 = -1$
 
 \item
The subgroup of ${\mathbb C}^\ast$ generated by $2i$
 
 \item
The subgroup of ${\mathbb C}^\ast$ generated by $(1 + i) / \sqrt{2}$
 
 \item
The subgroup of ${\mathbb C}^\ast$ generated by $(1 + \sqrt{3}\, i) / 2$
 
\end{enumerate}
 
 
\item
Find the subgroups of $GL_2( {\mathbb R })$ generated by each of the
following matrices. 
\begin{multicols}{3}
\begin{enumerate}
 
\item
$\displaystyle
\begin{pmatrix}
0 & 1 \\
-1 & 0
\end{pmatrix}
$

\item
$\displaystyle
\begin{pmatrix}
0 & 1/3 \\
3 & 0
\end{pmatrix}
$

\item
$\displaystyle
\begin{pmatrix}
1 & -1 \\
1 & 0
\end{pmatrix}
$

\item
$\displaystyle
\begin{pmatrix}
1 & -1 \\
0 & 1
\end{pmatrix}
$

\item
$\displaystyle
\begin{pmatrix}
1 & -1 \\
-1 & 0
\end{pmatrix}
$
 
\item
$\displaystyle
\begin{pmatrix}
\sqrt{3}/ 2 & 1/2 \\
-1/2 & \sqrt{3}/2
\end{pmatrix}
$

 \end{enumerate}
\end{multicols}

 

\item		  %%%%%%%%%%%%%%%%%%%%%%%%
Find the order of every element in ${\mathbb Z}_{18}$.
 
 
%% 6 %%%%%%%%%%%%%%%%%%%%%%%%%%%%%%%%%%%%%%%%%%%%%%%%
\item
Find the order of every element in the symmetry group of the square,
$D_4$.
 
 
\item
What are all of the cyclic subgroups of the quaternion group, $Q_8$? 
 
 
\item
List all of the cyclic subgroups of $U(30)$.
 
 
\item
List every generator of each subgroup of order 8 in ${\mathbb
Z}_{32}$.
 
% 2010/05/10 R Beezer: Added explanation of the * on sets
\item
Find all elements of finite order in each of the following groups. Here the ``$\ast$'' indicates the set with zero removed.
\begin{multicols}{3}
\begin{enumerate}
 
 \item
${\mathbb Z}$
 
 \item
${\mathbb Q}^\ast$
 
 \item
${\mathbb R}^\ast$
 
\end{enumerate}
 \end{multicols}
 
\item
If $a^{24} =e$ in a group $G$, what are the possible orders of $a$? 
 
 
\item
Find a cyclic group with exactly one generator.  Can you find cyclic
groups with exactly two generators?  Four generators?  How about $n$
generators?
 
 
\item
For $n \leq 20$, which groups $U(n)$ are cyclic?  Make a conjecture as
to what is true in general.  Can you prove your conjecture?  
 
 
\item
Let
\[
A=
\begin{pmatrix}
0 & 1 \\
-1 & 0
\end{pmatrix}
\qquad \text{and} \qquad
B=
\begin{pmatrix}
0 & -1 \\
1 & -1
\end{pmatrix}
\]
be elements in $GL_2( {\mathbb R} )$. Show that $A$ and $B$ have finite
orders but $AB$ does not. 
 
 
\item
Evaluate each of the following.
\begin{multicols}{2}
\begin{enumerate}
 
\item
$(3-2i)+ (5i-6)$

 
\item
 $(4-5i)-\overline{(4i -4)}$
 
 \item
$(5-4i)(7+2i)$
 
\item
$(9-i) \overline{(9-i)}$
 
 \item
$i^{45}$

\item
$(1+i)+\overline{(1+i)}$
 
\end{enumerate}
\end{multicols}
 
 \item   %%%%%%%%%%%%%%%%%%%%%%%%
Convert the following complex numbers to the form $a + bi$.
\begin{multicols}{2}
\begin{enumerate}

 \item
$2 \cis(\pi / 6 )$

 
 \item
$5 \cis(9\pi/4)$

\item
$3 \cis(\pi)$
 
 \item
$\cis(7\pi/4) /2$
 
\end{enumerate}
\end{multicols}


\item	  %%%%%%%%%%%%%%%%%%%%%%%%%%%%%%%%
Change the following complex numbers to polar representation.
\begin{multicols}{3}
\begin{enumerate}
 
 \item
$1-i$

 \item
$-5$
 
 \item
$2+2i$
 
 
\item
$\sqrt{3} + i$

 \item
$-3i$

 \item
$2i + 2 \sqrt{3}$
 
\end{enumerate}
\end{multicols}

 
 
\item %%%%%%%%%%%%%%%%%%%%%%%%
Calculate each of the following expressions.
\begin{multicols}{2}
\begin{enumerate}
 
 \item
$(1+i)^{-1}$

 \item
$(1 - i)^{6}$
 
 \item
$(\sqrt{3}+i)^{5}$

 \item
$(-i)^{10}$
 
 \item
$((1-i)/2)^{4}$

 \item
$(-\sqrt{2} - \sqrt{2}\, i)^{12}$
 
 \item
$(-2+2i)^{-5}$
 
\end{enumerate}
\end{multicols}

  
  \item
Prove each of the following statements.
\begin{multicols}{2}
\begin{enumerate}
 
 \item
$|z| = | \overline{z}|$

\item
$z \overline{z} = |z|^2$
 
 \item
$z^{-1} = \overline{z} / |z|^2$

 \item
$|z +w| \leq |z| + |w|$
 
 \item
$|z - w| \geq | |z| - |w||$
 
 \item
$|z w| = |z|  |w|$
 
\end{enumerate}
\end{multicols}


\item
List and graph the 6th roots of unity.  What are the generators of
this group?  What are the primitive 6th roots of unity?
 
 
\item
List and graph the 5th roots of unity.  What are the generators of
this group?  What are the primitive 5th roots of unity? 
 
 
  
\item
Calculate each of the following.
\begin{multicols}{2}
\begin{enumerate}
 
 \item
$292^{3171} \pmod{ 582}$

\item
$2557^{ 341} \pmod{ 5681}$

 \item
$2071^{ 9521} \pmod{ 4724}$
 
 \item
$971^{ 321} \pmod{ 765}$
 
\end{enumerate}
\end{multicols}
 
 
  
 
\item
Let $a, b \in G$.  Prove the following statements.
\begin{enumerate}
 
 \item
The order of $a$ is the same as the order of $a^{-1}$.
 
 \item
For all $g \in G$, $|a| = |g^{-1}ag|$.
 
 \item
The order of $ab$ is the same as the order of $ba$.
 
\end{enumerate}
 
 
\item
Let $p$ and $q$ be distinct primes.  How many generators does ${\mathbb
Z}_{pq}$ have? 
 
 
\item
Let $p$ be prime and $r$ be a positive integer.  How many generators
does ${\mathbb Z}_{p^r}$ have? 
 
 
\item
Prove that  ${\mathbb Z}_{p}$ has no nontrivial subgroups if $p$ is
prime. 
 
 
\item
If $g$ and $h$ have orders 15 and 16 respectively in a group $G$, what
is the order of $\langle g \rangle  \cap \langle h \rangle $? 
 
 
\item
Let $a$ be an element in a group $G$. What is a generator for the
subgroup $\langle a^m \rangle  \cap  \langle a^n \rangle $?
 
 
\item
Prove that ${\mathbb Z}_n$ has an even number of generators for $n > 2$. 
 
 
\item
Suppose that $G$ is a group and let $a$, $b \in G$. Prove that if $|a|
= m$ and $|b| = n$ with $\gcd(m,n) = 1$, then $\langle a \rangle  \cap
\langle b \rangle  = \{ e \}$. 
 
 
\item
Let $G$ be an abelian group. Show that the elements of finite order in
$G$ form a subgroup. This subgroup is called the \boldemph{torsion
subgroup}\index{Subgroup!torsion} of $G$. 
 
 
\item
Let $G$ be a finite cyclic group of order $n$ generated by $x$. Show
that if $y = x^k$ where $\gcd(k,n) = 1$, then $y$ must be a generator
of $G$.
 
 
\item
If $G$ is an abelian group that contains a pair of cyclic subgroups of
order 2, show that $G$ must contain a subgroup of order 4. Does this
subgroup have to be cyclic?
 
 
\item
Let $G$ be an abelian group of order $pq$ where $\gcd(p,q) = 1$.  If
$G$ contains elements $a$ and $b$ of order $p$ and $q$ respectively,
then show that $G$ is cyclic. 
 
 
\item
Prove that the subgroups of ${\mathbb Z}$ are exactly $n{\mathbb Z}$ for $n
= 0, 1, 2, \ldots$. 
 
 
\item
Prove that the generators of ${\mathbb Z}_n$ are the integers $r$ such
that $1 \leq r < n$ and $\gcd(r,n) =  1$. 
 
 
\item
Prove that if $G$ has no proper nontrivial subgroups, then $G$ is a 
cyclic group.
 
 
 
\item
Prove that the order of an element in a cyclic group $G$ must divide
the order of the  group. 
 
 
\item
For what integers $n$ is $-1$ an $n$th root of unity?
 
 
\item
If $z = r( \cos \theta + i \sin \theta)$ and $w = s(\cos \phi + i \sin
\phi)$ are two nonzero complex numbers, show that
\[
zw = rs[ \cos( \theta + \phi)  + i \sin( \theta + \phi)].
\]
 
 
\item
Prove that the circle group is a subgroup of  ${\mathbb C}^*$.
 
 
\item
Prove that the $n$th roots of unity form a cyclic subgroup of ${\mathbb
T}$  of order $n$. 
 
 
\item
Let $\alpha \in \mathbb T$. Prove that $\alpha^m =1$ and $\alpha^n = 1$ if and only if $\alpha^d = 1$
for $d = \gcd(m,n)$.

%Clarified the exercise by adding that $\alpha \in \mathbb T$.  Suggested by R. Beezer.
%TWJ - 2012/10/21
 
 
\item
Let $z \in {\mathbb C}^\ast$. If $|z| \neq 1$, prove that the order of
$z$ is infinite. 
 
 
\item
Let $z =\cos \theta + i \sin \theta$ be in ${\mathbb T}$ where $\theta
\in {\mathbb Q}$.  Prove that the order of $z$ is infinite.
 
\end{enumerate}
 
 
\subsection*{Programming Exercises}
 
 
{\small
\begin{enumerate}
 
 
\item
Write a computer program that will write any decimal number as the sum
of distinct powers of 2.  What is the largest integer that your
program will handle?
 
 
\item
Write a computer program to calculate $a^x \pmod{ n}$ by the method of
repeated squares.  What are the largest values of $n$ and $x$ that
your program will accept?  
 
 
\end{enumerate}
}
 
 
\subsection*{References and Suggested Readings}
 
 
{\small
\begin{itemize}
 
\item[\textbf{[1]}] %reference updated - TWJ 5/6/2010
Koblitz, N. \textit{A Course in Number Theory and Cryptography}. 2nd ed.
Springer, New York, 1994.  
 
 
\item[\textbf{[2]}]
Pomerance, C. ``Cryptology and Computational Number Theory---An
Introduction,'' in \textit{Cryptology and Computational Number Theory},
Pomerance, C., ed. Proceedings of Symposia in Applied Mathematics,
vol. 42, American Mathematical Society, Providence, RI, 1990.  This
book gives an excellent account of how the method of repeated squares
is used in cryptography. 
 
\end{itemize}
}
 

\end{document}
